%

% article class because we want to fully customize the page and not use a cv template
\documentclass[a4paper,12pt]{article}

%----------------------------------------------------------------------------------------
%	FONT
%----------------------------------------------------------------------------------------

% % fontspec allows you to use TTF/OTF fonts directly
% \usepackage{fontspec}
% \defaultfontfeatures{Ligatures=TeX}

% % modified for ShareLaTeX use
% \setmainfont[
% SmallCapsFont = Fontin-SmallCaps.otf,
% BoldFont = Fontin-Bold.otf,
% ItalicFont = Fontin-Italic.otf
% ]
% {Fontin.otf}

%----------------------------------------------------------------------------------------
%	PACKAGES
%----------------------------------------------------------------------------------------
\usepackage{url}
\usepackage{parskip} 	

%other packages for formatting
\RequirePackage{color}
\RequirePackage{graphicx}
\usepackage[usenames,dvipsnames]{xcolor}
\usepackage[scale=0.9]{geometry}

%tabularx environment
\usepackage{tabularx}

%for lists within experience section
\usepackage{enumitem}

% centered version of 'X' col. type
\newcolumntype{C}{>{\centering\arraybackslash}X} 

%to prevent spillover of tabular into next pages
\usepackage{supertabular}
\usepackage{tabularx}
\newlength{\fullcollw}
\setlength{\fullcollw}{0.47\textwidth}

%custom \section
\usepackage{titlesec}				
\usepackage{multicol}
\usepackage{multirow}

%CV Sections inspired by: 
%http://stefano.italians.nl/archives/26
\titleformat{\section}{\Large\scshape\raggedright}{}{0em}{}[\titlerule]
\titlespacing{\section}{0pt}{5pt}{4pt}

%for publications
\usepackage[style=authoryear,sorting=ynt, maxbibnames=2]{biblatex}

%Setup hyperref package, and colours for links
\usepackage[unicode, draft=false]{hyperref}
\definecolor{linkcolour}{rgb}{0,0.2,0.6}
\hypersetup{colorlinks,breaklinks,urlcolor=linkcolour,linkcolor=linkcolour}
\addbibresource{citations.bib}
\setlength\bibitemsep{1em}

%for social icons
\usepackage{fontawesome5}

%debug page outer frames
%\usepackage{showframe}

%----------------------------------------------------------------------------------------
%	BEGIN DOCUMENT
%----------------------------------------------------------------------------------------
\begin{document}

% non-numbered pages
\pagestyle{empty} 

%----------------------------------------------------------------------------------------
%	TITLE
%----------------------------------------------------------------------------------------

% \begin{tabularx}{\linewidth}{ @{}X X@{} }
% \huge{Your Name}\vspace{2pt} & \hfill \emoji{incoming-envelope} email@email.com \\
% \raisebox{-0.05\height}\faGithub\ username \ | \
% \raisebox{-0.00\height}\faLinkedin\ username \ | \ \raisebox{-0.05\height}\faGlobe \ mysite.com  & \hfill \emoji{calling} number
% \end{tabularx}

\begin{tabularx}{\linewidth}{@{} C @{}}
\Huge{Stefan de Lasa} \\[7.5pt]
\href{https://www.github.com/destefy/portfolio}{\raisebox{-0.05\height}\faGithub\ destefy } \ $|$ \ 
\href{https://www.linkedin.com/in/stefandelasa}{\raisebox{-0.05\height}\faLinkedin\ stefandelasa} \ $|$ \ 
\href{mailto:stefan.delasa@gmail.com}{\raisebox{-0.05\height}\faEnvelope \ stefan.delasa@gmail.com} \ $|$ \ 
\href{tel:+6479208916}{\raisebox{-0.05\height}\faMobile \ +1(647)-920-8916} \\
\end{tabularx}


% Education
\section{Education}
Faculty of Applied Science \& Eng., BASc - Computer Engineering, University of Toronto, 2020-2025. 
Seasonal GPA: 4.0 / 4.0, Cumulative GPA: 3.81 / 4.0, Dean’s Honour List.
    
% Skills
\section{Skills}
\begin{tabularx}{\linewidth}{@{}l X@{}}
Programming Languages: &  \normalsize{C++, C, Python, MATLAB, Javascript, Typescript, React, ARM Assembly}\\
Programming Tools:  &  \normalsize{Git, VSCode, IntelliJ, Jira, Bitbucket, Jenkins, Cypress}\\  
Hardware Tools: & \normalsize{Verilog, Multisim, ModelSim, Altium, Typhoon}
\end{tabularx}

% Experience
\section{Work Experience}

%PointClickCare
\begin{tabularx}{\linewidth}{ @{}l r@{} }
\textbf{Software Engineer Intern,} \textit{PointClickCare (PCC)} & \hfill May - Aug 2022, Toronto, ON \\[3.75pt]
\multicolumn{2}{@{}X@{}}{
PCC creates healthcare software solutions to assist vulnerable populations with out-of-hospital care.
\begin{minipage}[t][5.6cm]{\linewidth}
    \begin{itemize}[nosep,after=\strut, leftmargin=1em, itemsep=3pt]
    \item Migrated the US “Care Insights” application to Canadian markets.
        \begin{itemize}
        \item Configured a backend \textbf{Spring Boot} controller to determine session permissions via API calls. Permissions were then used across different application workflows (e.g., exposing links, etc.)
        \end{itemize}
    \item Worked in a 10 person \textbf{Agile development} team on a suite of applications for nursing facilities.
        \begin{itemize}
        \item Used \textbf{React} and \textbf{Typescript} to develop front-end features to ease creation/modification of patient screening templates. Several internal users mentioned improved usability from this work.
        \item To improve patient screening template effectiveness, I extracted session information about which end-user workflow suggestions were followed or ignored. I then sent this information to PCC’s Pendo analytics system for subsequent analysis and template refinement.
        \item Used \textbf{Cypress} and \textbf{Kotlin} to write service-level and unit tests to catch regressions and ensure front-end UI and data pipeline integrity.
        \end{itemize}
    \end{itemize}
    \end{minipage}
}  
\end{tabularx}

%IESO
\begin{tabularx}{\linewidth}{ @{}l r@{} }
\small{\textbf{Meter Data Management Intern,} \textit{Independent Electricity System Operator} & \small{\hfill Jun - Aug 2021, Toronto, ON} 
} \\[3.75pt]
\multicolumn{2}{@{}X@{}}{
As the Crown corporation responsible for operating/directing the electricity market in Ontario, the IESO gathers and monitors data from industrial customers throughout the province.
\begin{minipage}[t][2cm]{\linewidth}
    \begin{itemize}[nosep,after=\strut, leftmargin=1em, itemsep=3pt]
        \item Leveraged my technical knowledge to propose and conduct research into how \textbf{machine learning} could be used to improve existing processes. 
        \item Highlighted benefits of \text{supervised learning} to detect data anomalies using IESO’s historical datasets.
        \item Worked with peers to review Meter Service Provider data, ensuring correctness of meter billing reports.
    \end{itemize}
    \end{minipage}
}
\end{tabularx} \\

%Projects
\section{Projects}

\begin{tabularx}{\linewidth}{ @{}l r@{} }
\textbf{Radio Transceiver} & \hfill Jan - April 2022 -  \href{https://www.github.com/destefy/portfolio}{Link to Pictures} \\[3.75pt]
\multicolumn{2}{@{}X@{}}{
As part of my 2nd year Hardware Design class (ECE295), I worked on a Team of 3 students to design, build and test two radio transceiver (transmitter + receiver) components. Specific contributions included,
\begin{minipage}[t]{\linewidth}
    \begin{itemize} [nosep,after=\strut, leftmargin=1em, itemsep=3pt]
        \item Used \textbf{Altium} and \textbf{Multisim} to design the limiter, filter, mixer, and amplifier radio receiver circuits.
        \item Demonstrated successful integration of our subcircuits into a functioning radio receiver.
        \item Communicated the team’s design to technical and non-technical audiences through presentations
    \end{itemize}
    \end{minipage}
}
\end{tabularx}

\section{Other}
\begin{tabularx}{\linewidth}{@{}l X@{}}
Citizenship: &  \normalsize{Canadian and American}\\
Languages:  &  \normalsize{English (Native Proficiency), French (Native Proficiency), Polish (Beginner)}\\  
\end{tabularx}


\end{document}
