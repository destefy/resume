\begin{tabularx}{\textwidth}{ @{}l r@{} }
    \textbf{Software Development Intern}, \small{\textit{CentML,}} Toronto, ON & \hfill \textbf{Sept 2023 - Aug 2024 \href{https://github.com/CentML/centml-python-client}{\faLink}} \\[3.75pt]
    \multicolumn{2}{@{}X@{}}{
    \raggedright{
        CentML is creating tools to make machine learning (ML) inference and training more affordable and efficient. 
    } \\[3.75pt]
    \begin{minipage}[t]{\linewidth}
        \begin{itemize}
            \item Was the sole creator of the remote compilation project, building the infrastructure that allowed user's ML models to be compiled on a server. 
            Please click \href{https://github.com/CentML/centml-python-client}{here} to see the public \textbf{Python} client.
            \item Created both a \textbf{Kubernetes}-managed (\textbf{Dockerized}) and a locally-hosted version of the compilation server.
            \item Developped a method to uniquely and consistently hash various ML models, including \textbf{LLMs} and \textbf{CNNs}.
            \item Communicated with a \textbf{PostgreSQL} database and \textbf{AWS' S3} to upload user submitted models.
            \item Wrote integration tests the ensure the compilation server was correctly integrated into CentML's platform.
            % \item Used the \textbf{Python FastAPI} library to create a compilation server for users to compile ML models remotely.
        \end{itemize}
    \end{minipage}
    }
\end{tabularx}
\\[5pt]