\begin{tabularx}{\textwidth}{ @{}l r@{} }
    \textbf{Software Development Intern}, \small{\textit{CentML,}} Toronto, ON & \hfill \textbf{Sep 2023 - Aug 2024 \href{https://github.com/CentML/centml-python-client}{\faLink}} \\[3.75pt]
    \multicolumn{2}{@{}X@{}}{
    \raggedright{
        CentML develops tools to make machine learning (ML) inference and training more affordable and efficient. 
    } \\[3.75pt]
    \begin{minipage}[t]{\linewidth}
        \begin{itemize}
            \item Sole contributor of the remote compilation project, delivering distributed compilation of \textbf{Pytorch} ML models. 
            \item Created cloud-hosted infrastructure and software (\textbf{Python}) for a \textbf{Kubernetes}/\textbf{Docker} based remote compiler.
            \item Enabled users to host their own compilation server via a public \textbf{Python} client. Please click \href{https://github.com/CentML/centml-python-client}{here} to see it.
            \item Developed a method to uniquely and consistently hash ML models, including \textbf{LLMs} and \textbf{CNNs}.
            \item Used a \textbf{PostgreSQL} database and \textbf{AWS' S3} to cache compiled and uncompiled models.
            \item Wrote end-to-end tests to ensure the compilation server was correctly integrated into CentML's platform.
            % \item Used the \textbf{Python FastAPI} library to create a compilation server for users to compile ML models remotely.
        \end{itemize}
    \end{minipage}
    }
\end{tabularx}
\\[5pt]